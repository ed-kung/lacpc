\section{Results}\label{sec_results}

Table \ref{tab_ologit_results} reports the results from the ordered logit regression. To main coefficient of interest is the effect of semantic uniqueness on case outcome. Four specifications are presented which show the effect of progressively adding more explanatory variables to the regression. The coefficient on semantic uniqueness is robust to the gradual inclusion of more and more confounding variables. Importantly, by including suffix, semantic cluster, and council district fixed effects, we show that the effect of semantic uniqueness on case outcome is not driven by selection on case type or geography. 

%We estimate an ordered logit model to analyze the determinants of project outcomes. The dependent variable is increasing in approval status as proposals can be rejected or delayed, approved with conditions or modifications, or granted full approval. All models are estimated with robust standard errors. We progressively add fixed effects for semantic cluster, council district, and suffix groups to account for heterogeneity in proposal types, jurisdiction, and administrative classifications. Table \ref{tab_ologit_results} reports the results. 

The results suggest that evaluating proposals is cognitively demanding, and unfamiliar ones are harder to process. 
%We operationalize a proposal's recognizability through semantic uniqueness, a measure of how closely it resembles the kinds of cases commissioners typically encounter. Higher values indicate a less typical proposal. 
The coefficient on semantic uniqueness is negative and statistically significant across specifications, indicating that unfamiliar cases are less likely to be approved outright, and more likely to be approved with conditions or delayed or denied. In the full model, a one-standard-deviation increase in semantic uniqueness decreases the log odds of full approval by \gn{SemUniCoef} ($p<0.05$). By contrast, being placed on the consent calendar nearly doubles the odds of full approval (\gn{ConCalCoef}, $p<0.01$). These results demonstrate the measurable penalty that unfamiliar projects face from imposing higher cognitive costs on commissioners. Meanwhile, routine consent calendar items move through the process almost automatically. Other institutional factors such as agenda order, agenda length, and agenda perplexity are small and statistically insignificant, indicating that bureaucratic frictions may stem less from meeting logistics than from project familiarity or project typicality.

Commissioners also weigh the effects of opposition on reputational and political exposure. The results indicate that doubling the number of opposition letters decreases the log odds of approval category by \gn{NOppCoef} ($p<0.01$). In contrast, the coefficient on support letters is small and statistically insignificant. Commissioners appear systematically more sensitive to community opposition than to community support, perhaps reflecting a defensive orientation in their decision-making.

Table~\ref{tab_ologit_marginal_effects} reports the marginal effects, which can be interpreted as the predicted effect on outcome probability for a one unit change in the regressor, averaged over cases. 
%Moreover, the patterns are monotonic: outcomes worsen as semantic uniqueness and opposition increase, and improve across outcomes for consent calendar items. 
The results indicate that a one standard deviation increase in semantic uniqueness reduces the average probability of full approval by about \gn{SemUniME2} ($p<0.05$), while raising partial approval by about \gn{SemUniME1} ($p<0.05$) and denial/postponement by \gn{SemUniME0} ($p<0.05$). Likewise, for opposition letters, doubling their number lowers the average likelihood of approval by about \gn{NOppME2} ($p<0.01$), and increases partial approval by \gn{NOppME1} ($p<0.01$) and denial/postponement by \gn{NOppME0} ($p<0.01$). Placement on the consent calendar has the opposite effect. Full approval increases by about \gn{ConCalME2} ($p<0.01$), partial approval decreases by \gn{ConCalME1} ($p<0.01$), and denial/postponement decreases by \gn{ConCalME0} ($p<0.01$).

To assess the robustness of our findings to omitted variable bias, we follow the approach of benchmarking selection on unobservables against selection on observables \citep{AltonjiElderTaber2005}. If adding a rich set of controls barely shifts the coefficient of interest while substantially increasing explanatory power, then unobservables would have to be implausibly more important than observables to overturn the result. We use movements in coefficients and $R^2$ across nested specifications to compute the relative strength unobservables would need to eliminate the effect \citep{Oster2019}. Given standard benchmarks, our estimates are highly robust to concerns about omitted variable bias.

Overall, these results are consistent with our central argument that bureaucratic decision-making at the CPC reflects both bounded rationality and oversight risk. Unfamiliar projects are penalized because they require greater cognitive processing, while contested projects are penalized because they expose commissioners to greater political and reputational scrutiny. By contrast, routine and uncontested projects proceed with little resistance. These patterns highlight how the regulatory process itself shapes outcomes as a function of the trade-offs faced by commissioners.
