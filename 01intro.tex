\section{Introduction}\label{sec_intro}

There is growing recognition among researchers, policymakers, and even popular media that stringent land use regulations may be contributing to low housing production and high housing costs in America and in much of the Western world.\footnote{\citet{klein2025abundance} have recently brought the issue to popular attention in their book \emph{Abundance}.} In the urban economics literature, many papers have demonstrated a link between land use regulations and housing market outcomes. For example, \citet{ganongshoag2017} showed how rising house prices are correlated with housing supply regulations, and may have explained regional income divergence in the U.S. in recent decades. \citet{brueckner2020} showed how building heights in many U.S. cities are below competitive equilibrium levels due to floor-area-ratio restrictions. Similar findings have been demonstrated in different geographic markets: \citet{glaeser2009} in Boston, \citet{jackson2016} for California cities, \citet{hilber2016} in England. There are many more, more than can be listed here, and we direct the reader to \citet{gyourkomolloy2015} and \citet{molloy2020} for further review. In addition to the empirical effects of housing supply regulation on housing outcomes, a number of papers have argued that these effects can lead to spatial misallocation of resources \citep{turner2014, albouy2018, hsieh2019, gabriel2020}. 

Despite the growing recognition of the importance of land use regulations, still little is known about the regulatory production function itself. That is, we know that land use regulations affect outcomes---but how are the regulations produced and how are they enforced? And does the \emph{method} of production or enforcement meaningfully impact outcomes over and above the \emph{de jure} text of the regulations? For example, regulations will say what can be built, where, but often allow for exceptions. Even if exceptions are always granted, does the process of getting an exception meaningfully impact housing market outcomes? It seems likely that the answer is yes, as \citet{gabrielkung2025} showed that long bureaucratic approval times significantly contributes to the rate of housing production. Yet, little is known about the inputs and incentives of this bureaucratic process that researchers acknowledge as important.




