\section{Introduction}\label{sec_intro}

Since \citet{Simon1955}, the concept of bounded rationality has reshaped how we understand decision-making. Rather than assuming fully informed optimization, actors operate with limited attention, information, and computational capacity. Most decisions---indeed, most organizational arrangements---exist precisely to manage these limits. For an organization or a policy to function, it must economize on information, distributing attention and routinizing choice across actors \citep{CyertMarch1963}. Rules, hierarchies, and standards function as compression devices that simplify complexity so boundedly rational agents can act. Institutional design, in this sense, involves a trade-off: reducing informational complexity makes collective action possible, but oversimplification can constrain discretion and innovation.

Still, Simon's insight is more descriptive than operational. His original formulation was broad, leaving open what it means to ``satisfice'' or how cognitive constraints work in practice. Subsequent research has attempted to formalize bounded rationality through explicit limits on optimization or information processing. In one branch, agents are modeled as satisficers, engaging in constrained search over feasible options \citep{Stigler1961,Rubinstein1986,Salant2011}. Another tradition treats cognition as a scarce resource allocated across noisy or incomplete signals \citep{Radner1993,VanZandt1999,Sims2003,HebertWoodford2023}. Recent approaches impose restrictions on reasoning itself. Agents cannot process all states or clauses within a choice architecture and therefore use simplified mental representations \citep{ZhangLevin2017,Jakobsen2020}. Simulation studies extend these ideas to dynamic settings, showing that when decision-makers can only make local, incremental improvements \citep{Levinthal1997,MarengoDosi2005,Richters2021}, outcomes depend as much on the structure of the decision landscape as on individual preferences or rules.

These theoretical advances have added clarity, but translating them into empirical work remains challenging. Because cognitive limits are inherently unobservable, they can only be inferred indirectly through deviations from idealized rationality. Behavioral economists have modeled this process as probabilistic optimization, where agents make noisy or limited best responses \citep{McKelveyPalfrey1995,CamererHoChong2004}. Experimental studies have documented satisficing and cognitive constraints \citep{Guth2010,LimMatrosTurocy2014,AlaouiJanezicPenta2020}, but remain confined to stylized laboratory environments. Systematic evidence from field settings is scarce \citep{Kirman2010,FehrSchmidt1999}, leaving open how bounded rationality operates in practice. As a result, the empirical literature lacks large-scale, observational analyses that trace how cognitive constraints manifest in organizational decision environments.

This paper addresses that gap. We study bounded rationality in the empirical setting of the Los Angeles City Planning Commission (LA CPC), an administrative body that decides which development projects proceed in the city. Commissioners evaluate proposals that combine legal, architectural, and community considerations under tight time limits and public scrutiny. Each case requires filtering substantial information while balancing political and legal considerations. The public nature of the process generates a rich paper trail: every meeting produces agendas, minutes, and written comments that record both the information presented and the decisions made. We use these records to construct a measure of semantic uniqueness by converting the text into high-dimensional vector embeddings. Cases that use similar language cluster together, and a proposal's uniqueness is defined by how far it lies from the center of its cluster. Projects with greater semantic distance impose higher cognitive effort, as commissioners must interpret unfamiliar items.

At its core, the Commission's task reflects a classic principal-agent problem. Commissioners act as agents on behalf of the city, charged with advancing development while ensuring compliance with zoning rules and public expectations. For each proposal, commissioners decide whether to delay, modify, or approve a project, subject to two constraints. First, they face oversight risk: every decision is publicly scrutinized by residents, community organizations, and the press. Excessive leniency or excessive postponements can cause political backlash. We capture this dimension by applying sentiment analysis to public correspondence as an indicator of monitoring pressure. Second, commissioners face cognitive limits: each meeting includes dozens of cases that vary in legal and technical complexity. We proxy for the cognitive effort in interpreting proposals using semantic uniqueness, which measures how linguistically similar each proposal is relative to the kinds of cases the Commission typically encounters. 

We use this framework to estimate how cognitive limits and public oversight correspond to approval outcomes using an ordered logit specification. The results reveal a consistent pattern. Less familiar projects face a penalty: controlling for project type, council district, and case characteristics, a one-standard-deviation increase in semantic uniqueness reduces the log odds of full approval by about \gn{SemUniCoef}. Routine items, by contrast, move easily through the process. Placement on the Commission's consent calendar, which groups standard cases for joint approval, nearly doubles the probability of full approval. Public opposition exerts a strong negative influence---doubling the number of opposition letters reduces the log odds of approval by \gn{NOppCoef}---but the number of support letters provides little offsetting benefit. 

These dynamics are particularly consequential in the context of urban development, where regulatory decisions directly shape the pace and composition of housing supply. A large literature has shown that stringent land-use regulations constrain construction and contribute to high housing costs.\footnote{See, for example, 
\citet{glaeser2009},
\citet{hilber2016},
\citet{ganongshoag2017},
\citet{brueckner2020},
and \citet{gabrielkung2025}.
See \citet{gyourkomolloy2015} and \citet{molloy2020} 
for further reviews of the literature.} 
Yet much less is known about the \emph{regulatory production process} itself---how administrative capacity, attention, and procedural design influence which projects advance and which stall. By showing that cognitive and oversight constraints systematically affect approval outcomes, our analysis provides evidence on a key mechanism linking bureaucratic performance to housing supply.

Our study contributes to two related areas of research. First, the paper provides an empirical grounding for the theory of bounded rationality by documenting cognitive constraints in real organizational settings. Using policy documents, we extract and quantify information to capture latent cognitive dimensions such as familiarity, complexity, and attention. In doing so, the paper offers a general framework for quantifying the cognitive costs of decision-making, which can be applied across institutional contexts to examine how information-processing limits shape organizational behavior. Second, it extends models of bureaucratic behavior by treating regulatory outcomes as functions of cognitive capacities as well as incentives. Whereas prior work has focused on malformed incentives or political capture (e.g., \citealp{Prendergast2003}; \citealp{CarpenterKrause2012}; \citealp{BusuiocLodge2016}), our results show that even in the absence of explicit bias, limited processing capacity can produce variation in outcomes. The analysis further contributes to the literature on urban regulation and housing supply (e.g., \citealp{gyourkomolloy2015}; \citealp{molloy2020}) by suggesting that the pace of development depends not only on statutory constraints but also on the institutional capacities of the regulatory apparatus itself.

The remainder of the paper is organized as follows. Section \ref{sec_model} situates our analysis within the literature on bureaucratic decision-making and develops a simple model that formalizes the trade-off between oversight risk and cognitive cost. Section \ref{sec_data} describes the Los Angeles City Planning Commission and the construction of the text-based dataset. Section \ref{sec_methodology} outlines our empirical methodology and variables, including the computation of semantic uniqueness. Section \ref{sec_results} presents the main results and robustness checks, and Section \ref{sec_conclusion} concludes.