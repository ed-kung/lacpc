\section{Introduction}\label{sec_intro}

There is growing recognition among researchers, policymakers, and even popular media that stringent land use regulations are contributing to reduced housing production and high housing costs in America and much of the Western world.\footnote{\citet{klein2025abundance} have recently brought the issue to popular attention in their book \emph{Abundance}.} In the urban economics literature, many papers have demonstrated a link between land use regulations and housing market outcomes. For example, \citet{ganongshoag2017} showed how rising house prices are correlated with housing supply regulations, and may have explained regional income divergence in the U.S. in recent decades. \citet{brueckner2020} showed how building heights in many U.S. cities are below competitive equilibrium levels due to floor-area-ratio restrictions. Similar findings have been demonstrated in different geographic markets: \citet{glaeser2009} in Boston, \citet{jackson2016} for California cities, \citet{hilber2016} in England. There are many more, more than can be listed here, and we direct the reader to \citet{gyourkomolloy2015} and \citet{molloy2020} for further review. In addition to the empirical effects of housing supply regulation on housing outcomes, a number of papers have argued that these effects can lead to spatial misallocation of resources \citep{turner2014, albouy2018, hsieh2019, gabriel2020}. 

Despite the growing understanding of the impact of land use regulations, still little is known about the regulatory production function itself. That is, we know that land use regulations affect outcomes---but how are the regulations produced and how are they enforced? And does the regulatory \emph{process} itself meaningfully impact outcomes over and above the \emph{de jure} text of the regulations? For example, regulations say what can be built, where, but often allow for exceptions. Even if exceptions are ultimately always granted, does limited capacity for processing exceptions impact housing outcomes? It seems likely that the answer is yes, as \citet{gabrielkung2025} showed that long bureaucratic approval times can significantly reduce to the rate of housing production. Yet, little is known about the role of bureaucratic capacity or its determinants.

This paper aims to fill that gap by providing empirical evidence for the role of bureaucratic capacity in the housing production process. We use data from the proceedings of the Los Angeles City Planning Commission (LA CPC). The LA CPC is a nine member administrative body with members appointed by the Mayor of LA and confirmed by the City Council. The CPC is responsible for reviewing zoning changes, approving conditional use permits for large developments, and handling appeals of decisions made by lower bodies. Decisions of the CPC are made by majority rule. The agenda for each meeting is set by the Commission itself and made available to the public at least 7 days before the meeting. The minutes of each meeting are also published online. The minutes record the order of discussion, the motions made on each agenda item, and the votes made on each motion. In addition, members of the public are allowed to submit comments on any agenda items either in writing or in person during the meeting.

The agenda, minutes, and written public comments of each CPC meeting are all made available to the public by the City of Los Angeles. We use this data to investigate the determinants of CPC decision making. We find that the CPC is responsive to public opinion: greater public opposition to a project makes it less likely to be approved, and more likely to get denied, delayed, or to have conditions attached to the approval. In addition, we find that unusual projects, as measured by their semantic dissimilarity to other requests, are also more likely to be denied, delayed, or to have conditions attached to the approval. This suggests that bounded rationality may play a role in the approvals process: projects that are unfamiliar and fall outside of normal routines require more cognitive load from the Commission, and are thus less likely to pass through smoothly. The statistical relationship between case approval and semantic uniqueness survives even after controlling for potential confounding factors, such as the types of requests made in the proposal and the overall complexity of the proposal. Lastly, we find that the CPC uses a procedural tool known as a ``Consent Calendar'' to streamline approvals for routine and non-controversial items, further consistent with the hypothesis that cognitive capacity is an important input in the CPC's bureaucratic decision-making.

To our knowledge, this is the first paper to provide direct empirical evidence of the bureaucratic inputs to the land use regulatory process. We demonstrate that public opposition and cognitive capacity are both significant drivers of LA CPC decision-making. Although we do not quantify the impact of these factors on aggregate housing supply, the results do suggest that reducing regulatory complexity and streamlining the regulatory process could lead to a meaningful acceleration of housing production. The results also contribute to the broader literature on bureaucratic decision-making and bounded rationality, which we discuss further in Section \ref{sec_model}.

The rest of the paper is organized as follows. Section \ref{sec_model} discusses the literature on bureaucratic decision-making and develops a simple model of CPC decision-making that can rationalize our empirical results. Section \ref{sec_data} describes how we collected and processed the data. Section \ref{sec_methodology} describes the empirical methodology and section \ref{sec_results} presents the results. Section \ref{sec_conclusion} concludes.



