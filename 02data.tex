\section{Data}\label{sec_data}

\subsection{Institutional Background}

The Los Angeles planning and approvals process for urban development is a multi-layered process that requires the input of multiple agencies. 


\subsection{Data Acquisition}

The Planning Department's website maintains robust public documentation of City Planning Commission Meetings. For each meeting, the agenda, minutes, and any supplemental documents relevant to the meeting (letters from the public, traffic assessments, architectural reports, etc.) are available for download as PDF files.\footnote{As of August 26th, 2025, these documents are available at the URL: \url{https://planning.lacity.gov/about/commissions-boards-hearings}.} We downloaded these documents for all City Planning Commission meetings from May 10th, 2018 to December 19th, 2024. This resulted in documentary data for \gn{NumberOfMeetings} meetings, covering \gn{NumberOfAgendaItems} agenda items, with \gn{NumberOfSupplementalDocs} supplemental documents, spanning \gn{PageCount} pages of PDF documents. Download occurred on April 10th, 2025.

Since we are primarily interested in bureaucratic decision making, we limit our attention to the agenda items requiring a decision from the board. These are identified by agenda items titled by their Planning Department case numbers, which have a standardized format of ``[CASE PREFIX]-[YEAR]-[SERIAL NUMBER]-[CASE SUFFIXES]''. Other agenda items include items like ``Director's Report'' and ``General Public Comment'' which do not require any decisions on the part of the board. Altogether, there were \gn{NumberOfCases} agenda items requiring a decision from the board, as identified by their Planning Department case numbers.

A typical agenda item is a request from a developer to approve a development plan that goes beyond what the site's zoning designation would allow, or an appeal of a previously approved plan. Figure \ref{fig_example_agenda_item} shows an example of an agenda item. The case number is DIR-2019-6048-TOC-SPR-WDI-1A. This was a project that was initially approved by the Director of Planning (DIR), granting the project bonuses under the Los Angeles Transit Oriented Communities program (TOC), requiring a Site Plan Review (SPR), and granting a waiver of dedication and improvements (WDI). However, this previously approved plan was appealed (1A), and the appeal is now to be considered by the City Planning Commission. In addition to the information contained in the case number, the agenda shows additional information such as the Countil District that the project is located in, and other specific details about the project proposal.

Figure \ref{fig_example_minutes_item} shows the associated minutes for the agenda item shown in Figure \ref{fig_example_agenda_item}. From the minutes, we can see that the appeal was granted in part and denied in part. The Director of Planning's previous decision was upheld by the CPC, but additional conditions were applied, thus allowing the project to move forward as long as the developer adheres to the new conditions. This outcome, the partial granting of an appeal or the approval of a project with modifications, is common but not the only kind of outcome. Sometimes, the requested actions by the developer are granted in their entirety. Rarely, the requested actions are denied entirely. A more common occurrence than denial is that the CPC puts the decision off to a later date. We will discuss the distribution of motion outcomes and voting patterns later in Section \ref{sec_descriptive_statistics}.

Figures \ref{fig_example_support_letter} and \ref{fig_example_oppose_letter} show examples of letters submitted by the public in support of and in opposition to the above project. The support letter emphasizes how the project will ease traffic, reduce air pollution, and increase housing availability. The opposition letter emphasizes concerns about displacement and how the proposed units will be unaffordable to current residents of the neighborhood. These letters typify the kinds of concerns expressed by community residents in this dataset; however, the number of letters this project attracted is atypical---it turns out that DIR-2019-6048 was a particularly controversial project. We will discuss the distribution of support and oppose letters across projects later in Section \ref{sec_descriptive_statistics}.

\subsection{Data Extraction}

The documentary data provides a wealth of information about CPC cases and their outcomes. However, the information is locked within textual data that is difficult to process using traditional methods. For example, traditional NLP methods based on token and pattern matching would have a hard time comparing the agenda to the minutes and determining whether the requested actions were approved, partially approved, approved with conditions, denied, or whether the decision was postponed to a future date. 

To extract usable features from the textual content more robustly, we make use of OpenAI's \texttt{gpt-4o} generative chat model. For example, the model can be asked to read the text of the agenda, read the text of the minutes, and explain what the result of committee's proposed motion was in terms of its implications for the development project: was the project approved, partially approved or approved with conditions, denied, or was the decision postponed? The methodology and prompts we used to perform the data extraction are described in detail in Appendix \ref{sec_data_appendix}. In this section, we will instead focus on what data features were extracted form the text.

\paragraph{Agenda items.} For each agenda item, extract the following information from its agenda text: 
\begin{itemize}
\item Item number (used primarily for identification);
\item Item title (for cases requiring a decision, this is always a Planning Department case number)
\item Short AI-generated summary of the agenda item's content;
\item List of related cases;
\item The Council District(s) to which the item applies;
\item The last day to act;
\item List of referenced laws, ordinances, or programs;
\item Indicator for whether the item is an appeal of a previous decision.
\end{itemize}

\paragraph{Minutes.} For each agenda item, we extract the following information from its minutes text:
\begin{itemize}
\item Short AI-generated summary of the deliberations and the motion that was ultimately voted on;
\item Name of Commission Member who moved the motion;
\item Name of the Member who seconded;
\item Names of the Members who voted aye, voted nay, abstained, were recused, or were absent;
\item Result of the vote (whether the motion passed or failed\footnote{Note that a motion passing is not the same as a project getting approved, nor is a motion failing the same as a project getting denied. A Member may move to deny the project's requested actions, or move to accept an appeal of a previously approved project, in which case the motion passing implies a denial of the project. These nuances highlight why LLMs are helpful in the data extraction process.});
\item Implication for the proposed project: whether the requested actions were approved, approved in part or with modifications, denied, or whether the deliberations were continued to a future date.
\end{itemize}

\paragraph{Supplemental documents.} For each supplemental document, we extract the following information from its text:
\begin{itemize}
\item Type of document: whether it is a letter or petition, a technical modification or procedural matter, a scientific or technical report (traffic, environmental, etc), or a credentials document (CV, resume, biography, etc); 
\item Type of author: whether the author of the document is an individual, an advocacy group, a consultant, a lawyer, a developer, or a public official;
\item Which agenda item(s) it references;
\item Short AI-generated summary of the document contents;
\item Support or opposition: Whether the document definitely supports, somewhat supports, is neutral towards, somewhat opposes, or definitely opposes the referenced agenda item(s).
\end{itemize}

\subsection{Descriptive Statistics} \label{sec_descriptive_statistics}

The resulting dataset contains \gn{NumberOfCases} agenda items. The distribution of project outcomes is shown in Table 
