\section{Data}\label{sec_data}

\subsection{Institutional Background}

The Los Angeles planning and approvals process for urban development is a multi-layered process that requires the input of multiple agencies. 

\subsection{Data Acquisition}

The Planning Department's website maintains robust public documentation of City Planning Commission Meetings. For each meeting, the agenda, minutes, and any supplemental documents relevant to the meeting (letters from the public, traffic assessments, architectural reports, etc.) are available for download as PDF files.\footnote{As of August 26th, 2025, these documents are available at the URL: \url{https://planning.lacity.gov/about/commissions-boards-hearings}.} We downloaded these documents for all City Planning Commission meetings from May 10th, 2018 to December 19th, 2024. This resulted in documentary data for \gn{NumberOfMeetings} meetings, covering \gn{NumberOfAgendaItems} agenda items, with \gn{NumberOfSupplementalDocs} supplemental documents, spanning \gn{PageCount} pages of PDF documents. Download occurred on April 10th, 2025.

Since we are primarily interested in bureaucratic decision making, we limit our attention to the agenda items requiring a decision from the board. These are identified by agenda items titled by their Planning Department case numbers, which have a standardized format of ``[CASE PREFIX]-[YEAR]-[SERIAL NUMBER]-[CASE SUFFIXES]''. Other agenda items include items like ``Director's Report'' and ``General Public Comment'' which do not require any decisions on the part of the board. Altogether, there were \gn{NumberOfCases} agenda items requiring a decision from the board, as identified by their Planning Department case numbers.

A typical agenda item is a request from a developer to approve a development plan that goes beyond what the site's zoning designation would allow, or an appeal of a previously approved plan. Figure \ref{fig_example_agenda_item} shows an example of an agenda item. The case number is DIR-2019-6048-TOC-SPR-WDI-1A. This was a project that was initially approved by the Director of Planning (DIR), granting the project bonuses under the Los Angeles Transit Oriented Communities program (TOC), requiring a Site Plan Review (SPR), and granting a waiver of dedication and improvements (WDI). However, this previously approved plan was appealed (1A), and the appeal is now to be considered by the City Planning Commission. In addition to the information contained in the case number, the agenda shows additional information such as the Countil District that the project is located in, and other specific details about the project proposal.

Figure \ref{fig_example_minutes_item} shows the associated minutes for the agenda item shown in Figure \ref{fig_example_agenda_item}. From the minutes, we can see that the appeal was granted in part and denied in part. The Director of Planning's previous decision was upheld by the CPC, but additional conditions were applied, thus allowing the project to move forward as long as the developer adheres to the new conditions. This outcome, the partial granting of an appeal or the approval of a project with conditions, is common but not the only kind of outcome. Sometimes, the requested actions by the developer are granted in their entirety. Sometimes, the requested actions are denied entirely but this is rare. A more common occurrence is that the CPC puts the decision off to a later date. We will discuss the distribution of motion outcomes and voting patterns later in Section XX.

Figures \ref{fig_example_support_letter} and \ref{fig_example_oppose_letter} show examples of letters submitted by the public in support of and in opposition to the above project. The support letter emphasizes how the project will ease traffic, reduce air pollution, and increase housing availability. The opposition letter emphasizes concerns about displacement and how the proposed units will be unaffordable to current residents of the neighborhood. These letters typify the kinds of concerns expressed by community residents in this dataset; however, the number of letters this project attracted is atypical---it turns out that DIR-2019-6048 was a particularly controversial project. We will discuss the distribution of support and oppose letters across projects later in Section XX.

\subsection{Data Extraction}

The documentary data provides a wealth of information 



