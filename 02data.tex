\section{Data}\label{sec_data}

\subsection{Institutional Background}

The Los Angeles planning and approvals process for urban development is a multi-layered process that requires the input of multiple agencies. 

\subsection{Data Acquisition}

The Planning Department's website maintains robust public documentation of City Planning Commission Meetings. For each meeting, the agenda, minutes, and any supplemental documents relevant to the meeting (letters from the public, traffic assessments, architectural reports, etc.) are available for download as PDF files.\footnote{As of August 26th, 2025, these documents are available at the URL: \url{https://planning.lacity.gov/about/commissions-boards-hearings}.} We downloaded these documents for all City Planning Commission meetings from May 10th, 2018 to December 19th, 2024. This resulted in documentary data for \gn{NumberOfMeetings} meetings, covering \gn{NumberOfAgendaItems} agenda items, with \gn{NumberOfSupplementalDocs} supplemental documents, spanning \gn{PageCount} pages of text. Download occurred on April 10th, 2025.

Since we are primarily interested in bureaucratic decision making, we limit our attention to the agenda items requiring a decision from the board. These are identified by agenda items titled by their Planning Department case numbers, which have a standardized format of ``[CASE PREFIX]-[YEAR]-[SERIAL NUMBER]-[CASE SUFFIXES]''. Other agenda items include items like ``Director's Report'' and ``General Public Comment'' which do not require any decisions on the part of the board. Altogether, there were \gn{NumberOfCases} agenda items requiring a decision from the board, as identified by their Planning Department case numbers.

A typical agenda item is a request from a developer to approve a development plan that goes beyond what the site's zoning designation would allow, either in terms of the architectural properties of the development or in terms of the proposed use of the development. Figure \ref{fig_example_agenda_item} shows an example of an agenda item. Its case number is CPC-2023-6389-CU-DB-WDI-HCA-PHP, indicating that it is a case to be decided by the CPC, submitted in 2023, assigned a serial number of 6389, is requesting a conditional use permit (CU), a density bonus (DB), a waiver of dedication and improvements (WDI), is eligible for benefits under the California SB 330 Housing Crisis Act (HCA), and is assigned Priority Housing Project status (PHP). In addition to the information contained in its case number, the agenda shows additional information such as the Council District that the project is located in, and specific details about the project proposal. 

Figure \ref{fig_example_minutes_item} shows the associated minutes for the agenda item shown in Figure \ref{fig_example_agenda_item}. From the minutes, we can see that all the requested actions were granted. This is not always the case, as sometimes only some of the requested actions are granted while others are denied; other times, the requested actions are granted, but with modifications; and still other times, the requests are denied entirely or the decision is put off to a later date. From the minutes, we also observe which Commission Members voted for or against the proposal. In this case, the affirmative votes were unanimous, with two Commission Members being absent. We will discuss the distribution of motion outcomes and voting patterns later in Section XX.





