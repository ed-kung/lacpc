\section{Conclusion}\label{sec_conclusion}

This paper examined how cognitive and oversight constraints shape bureaucratic decision-making in the Los Angeles City Planning Commission. We developed a text-based measure of semantic uniqueness to capture the cognitive effort that unfamiliar proposals impose on commissioners, and used sentiment analysis of public correspondence to measure monitoring pressure. The results reveal that both constraints have significant effects on regulatory outcomes. Less familiar projects, which require greater interpretive effort, are less likely to receive full approval, while routine items proceed with little resistance. Public opposition exerts a similarly negative effect, whereas public support provides little offsetting influence. These patterns provide empirical evidence of bounded rationality in administrative decision-making, with bureaucrats allocating scarce attentional resources across proposals, prioritizing those that are familiar or safe under scrutiny. Institutional mechanisms such as the consent calendar function help economize on attention.

The results also have implications for urban economics. Land-use regulation remains one of the central bottlenecks in housing supply, yet most of the literature focuses on statutory rules rather than the bureaucratic process that enforces them. Our analysis shows that, beyond formal regulation, administrative capacity itself constrains housing production. Negative public pressure clearly shapes outcomes, but cognitive limits also constrain approval of novel or complex projects. The result is a form of institutional lock-in. By favoring projects that are easiest to interpret and defend, the process reinforces its own expectations, narrowing the range of designs it can evaluate. The outcome may be an urban landscape optimized for administrative legibility rather than public value.

More broadly, our framework demonstrates how text can be used to represent the decision architecture within organizations. With such data becoming increasingly available, semantic uniqueness provides a scalable measure to study how information is processed, attention is allocated, and discretion is transformed into routines. These mechanisms deepen our understanding of how organizations adapt, and the conditions under which they resist change.


\pagebreak