\section{Introduction}\label{sec_intro}

Modern cities struggle to build housing at the scale required to sustain growth. Rising housing costs have become one of the central economic challenges facing cities in the United States and much of the Western world.\footnote{\citet{klein2025abundance} have recently brought the issue to popular attention in their book \emph{Abundance}.} A large body of work in urban economics documents that stringent land-use regulations substantially constrain housing supply, contributing to high prices, reduced construction, and spatial misallocation of labor and capital (e.g., \citealp{glaeser2009, hilber2016, ganongshoag2017, brueckner2020}; see \citealp{gyourkomolloy2015} and \citealp{molloy2020} for reviews). The implications extend well beyond housing markets, shaping regional inequality, migration patterns, and aggregate productivity.

Despite this growing literature, most empirical work treats regulation as a \textit{static constraint}---a set of zoning rules, density limits, or statutory requirements that determine what can be built and where. Much less is known about the \textit{regulatory production process} itself---how development proposals are evaluated, modified, delayed, or approved by administrative bodies, and how this process affects urban outcomes over and above the formal text of land-use regulations. Yet in practice, housing supply is mediated through discretionary review and appeals, which require administrative processing. Even when projects ultimately comply with zoning rules or are eventually approved, conditions introduced during the administrative process can slow the pace of construction. Recent evidence suggests that prolonged approval timelines alone can significantly depress housing production (\citealp{gabrielkung2025}), highlighting the importance of bureaucratic capacity as a potential bottleneck in urban development.

This paper studies the regulatory production process in the context of urban land use. We examine decision-making by the Los Angeles City Planning Commission (LA CPC), the administrative body responsible for approving large development projects in the City of Los Angeles. Commissioners act on behalf of the city, charged with advancing development while ensuring compliance with zoning rules and public expectations. They evaluate proposals combining legal, architectural, and community considerations under tight time constraints. The public nature of the process generates a rich paper trail: every meeting produces agendas, minutes, and written comments that record both the information presented and the decisions made. These records provide a window into the regulatory production process.

At its core, the LA CPC's task reflects a classic principal-agent problem shaped by two binding constraints. First, commissioners face oversight risk. Decisions are scrutinized by residents, community organizations, and the media, and public opposition can trigger political backlash. We capture this dimension by applying sentiment analysis to public correspondence as an indicator of monitoring pressure. Second, they face cognitive constraints. Development proposals vary widely in complexity and novelty, and evaluating unfamiliar cases requires greater interpretive effort. When administrative attention is scarce, routine or familiar projects may be easier to process than idiosyncratic ones. We capture this dimension by constructing a measure of semantic uniqueness that quantifies how atypical a development proposal is relative to other cases. Using high-dimensional text embeddings of agenda items, we cluster proposals into comparable categories and measure how far each project lies from the center of its cluster. Projects with greater semantic distance are less familiar and impose higher cognitive demands on commissioners. 

We use this framework to estimate how oversight risk and cognitive constraints relate to approval outcomes using an ordered logit specification. The results reveal a systematic pattern. Less familiar projects face a penalty: controlling for project type, council district, and other case characteristics, a one-standard-deviation increase in semantic uniqueness reduces the log odds of full approval by about \gn{SemUniCoef}. Public opposition matters as well: doubling the number of opposition letters reduces the log odds of approval by \gn{NOppCoef}, while support letters provide little offsetting benefit. In contrast, routine projects move more smoothly through the process. Placement on the Commission's consent calendar--a procedural device that bundles standard, non-controversial items for joint approval--nearly doubles the probability of full approval. These findings provide evidence on a key mechanism through which bureaucratic performance affects housing supply. Regulatory outcomes are shaped not only by statutory rules or political incentives, but also by limits on administrative capacity to process information at scale. In this setting, bureaucratic capacity acts as an implicit constraint on development, reinforcing delay and discouraging innovation in project design. 

The paper contributes to two strands of the literature. First, it extends research in urban economics on land-use regulation and housing supply by opening the ``black box'' of the regulatory production process. We show that the pace of development depends not only on statutory constraints but also on the administrative capacities of the regulatory apparatus itself. Second, the paper contributes to the literature on bureaucratic decision-making by providing field-based evidence of bounded rationality in an organizational setting. Rather than treating cognitive limits as an abstract behavioral assumption, we measure familiarity directly from policy documents and show how it shapes real regulatory outcomes.

The remainder of the paper proceeds as follows. Section \ref{sec_model} presents a simple model of bureaucratic decision-making that formalizes the trade-off between cognitive effort and oversight risk. Section \ref{sec_data} describes the Los Angeles City Planning Commission and the construction of the text-based dataset. Section \ref{sec_methodology} outlines the empirical strategy and variables, including the computation of semantic uniqueness. Section \ref{sec_results} presents the main results and robustness checks, and Section \ref{sec_conclusion} concludes.


